\chapter{Methods and algorithms}
\section{Langevin equation}
First in general, providing path integral solutions, eq. distribution ecc. Then do the specific case for a scalar theory. Then add fermions and discuss the tracelog contribution, mentioning stochastic trace evaluation. \\
Scalar theory:
\begin{equation}
    \partial_\tau \phi(\tau, x) = - \frac{\delta S[\phi]}{\delta \phi (\tau, x)} + \eta (\tau, x)
    \label{eq:Langevin_scalar_full}
\end{equation}
At equilibrium the probability of a field configuration $\phi$ reads
\begin{equation}
    \mathcal{P}(\phi) = \frac{1}{Z} \, \exp\left(-S[\phi])\right)
    \label{eq:probability_field_configuration_full}
\end{equation}
This allow one to compute correlation functions as moments of the distribution \eqref{eq:probability_field_configuration_full}.



\section{Coloured noise}
\label{sec:coloured_noise}
\note{Mention some possible uses of colored noise beside taking continuum limits.} \\
In the stochastic quantisation procedure the noise which accounts for the quantum fluctuations of the theory is assumed to be white noise. This means that its power spectrum is flat in momentum space, extending in all the first Brillouin zone, namely for $p_\mu \ in [-\pi/a, \pi/a]$. One could modify this spectrum, and in this case one says \emph{colored noise}. In particular one could put a sharp cutoff on the total momentum, imposing $p^2 \leq \Lambda^2$. We refer to this particular case as \emph{regularised noise}. An example for the one-dimensional case is reported in figure \ref{fig:white_noise_colored_noise}.
\begin{figure}
    \centering
    \includegraphics[scale=0.7]{figures/colorednoise.png}
    \caption{Regularised noise}
    \label{fig:white_noise_colored_noise}
\end{figure}
In such case the Langevin equation for the scalar field \eqref{eq:Langevin_scalar_full} assumes the form
\begin{equation*}
    \partial_\tau \phi(\tau, x) = - \frac{\delta S[\phi]}{\delta \phi (\tau, x)} + r_\Lambda (x) \, \eta (\tau, x)
    \label{eq:Langevin_scalar_regularised}
\end{equation*}
where the regularising function $r_\Lambda(x)$ can be easilly expressed in momentum space as $r_\Lambda(p) = \theta(\Lambda^2 - p^2)$. One can show \cite{Pawlowski2017CoolingNoise} that the stochastic process is now driven towards a new equilibrium distribution
\begin{equation}
    \mathcal{P}_\Lambda(\phi) = \frac{1}{Z} \, \exp\left(-S_\Lambda[\phi])\right) = \frac{1}{Z} \, \exp\left(-(S[\phi] + \Delta S_\Lambda[\phi])\right)
    \label{eq:probability_field_configuration_regularised}
\end{equation}
where the correction term $S_\Lambda[\phi]$ ensures that the probability measure $\mathcal{P}_\Lambda$ vanishes for squared field's momenta greater than the cutoff $\Lambda^2$. \\
An explicit example of such regulator for a free scalar field can be ADD EQUATION PAWLOWSKI.
This is just one example of regularisation and different ones can be chosen. See \cite{Pawlowski2017CoolingNoise} for details. \\
Note that the formulation presented in this section includes also the NJL-model as a particular case, after integrating out the fermionic fields as described in section \ref{sec:lattice_discretisation}.


\section{Lattice QFT with regularised noise}
After the general introduction on coloured noise given in the previous paragraph, let us now look more closely on the lattice formulation and at the various applications of such techniques.
From a code perspective, the algorithm to regularise noise with a sharp cutoff is presented in Appendix WHICH ONE?. \\
Let us consider a squared two-dimensional lattice with size $L \equiv L_x = L_y$ and spacing $a \equiv a_x = a_y$. This implies a maximum momentum $p_\text{max} = \pi / a$ in each space-time direction. We consider a regularised simulation with cutoff $\Lambda'$. Let us also define $\Lambda^2 \equiv (p^x_\text{max})^2 + (p^y_\text{max})^2$. At this point we introduce a parameter, called \emph{cutoff fraction}, defined as $s^2 \equiv \Lambda^2 / \Lambda'^{\,2}$. With this notation, $s=1$ is the full white noise case, while for any regularised noise one has $0 < s < 1$. \\
We now want to address the following question: given the stationary probability distribution \ref{eq:probability_field_configuration_regularised}, is it possible to compensate the change in physical observables caused by the removal of the quantum modes via regularised noise, by a rescaling of the bare parameters that enter the lattice discretised action? Remembering from sections \ref{sec:lattice_discretisation} and \ref{sec:continuum_limit} that $\Lambda \sim a^{-1}$, this question would also address the problem of continuum limit of effective field theories. In fact the question we want to address is completely equivalent to the following: can one compensate a change in the spacing (controlled by the bare parameters), by a change of the noise in the simulation, in order to keep physical observables constant?\\
To this end, let us consider a noise regularisation given by $s^2 = \Lambda^2 / \Lambda'^{\,2} < 1$ ($s = a' / a < 1$), which means a higher cutoff (a smaller spacing $a'$). We now introduce an approximate ansatz which is based on the analogy to standard block-spin transformations. Higher order corrections to this simple ansatz are discussed in WHICH SECTION? \\
A change in the spacing $a \to ra$ will cause the following
\begin{gather*}
    \hat m_0^2 = (a m_0)^2 \to (ra m_0)^2 = r^2 (a m_0)^2 = r^2 \hat m_0^2 \\
    \Lambda \to \Lambda / r
\end{gather*}

Not that after rescaling, the Dirac operator transforms as $D \to s D$. This means that the full action transforms as $S = S_\phi + \text{Tr} \log D \to S' = S_\phi + s^2 \text{Tr} \log D + s^2 \text{Tr} \log s$.
The last term is field independent, hence one could simply drop it and the equations of motion would remain the same. Equivalently, the Langevin drift contribution of that term is zero, hence it does not contribute to the dynamics. This means in particular that one does not have to rescale the couplings that enter only in the Dirac operator, such as $m_q$ and $g$.

\begin{table}[]
    \centering
    \begin{tabular}{cccccccc}
        \toprule
         $s$ & $a$ & $\Lambda$ & $N$ & $m_\phi^2$ & $\lambda$ & $m_{q, 0}$ & $g_0$  \\
         \midrule 
         $1$ & $a$ & $\Lambda$ & $8$ & $m_{\phi, 0}^2$ & $\lambda$ & $m_{q, 0}$ & $g_0$ \\
         $2$ & $a/2$ & $2\Lambda$ & $16$ & $m_{\phi, 0}^2 / 4$ & $\lambda / 4$ & $m_{q, 0}$ & $g_0$ \\
         $4$ & $a/4$ & $4\Lambda$ & $32$ & $m_{\phi, 0}^2 / 16$ & $\lambda / 16$ & $m_{q, 0}$ & $g_0$ \\
         $8$ & $a/8$ & $8\Lambda$ & $64$ & $m_{\phi, 0}^2 / 64$ & $\lambda / 64$ & $m_{q, 0}$ & $g_0$ \\
         \bottomrule
    \end{tabular}
    \caption{Rescaling}
    \label{block_spin_steps}
\end{table}

\newpage

\begin{figure}[h]
\centering
\begin{minipage}{0.45\textwidth}
    \begin{tikzpicture}
        \begin{axis} [axis lines=center, xtick=\empty, ytick=\empty, xlabel=$\phi$, ylabel=$V(\phi)$,
        every axis x label/.style={
            at={(ticklabel* cs:1.0)},
            anchor=west,
        },
        every axis y label/.style={
            at={(ticklabel* cs:1.0)},
            anchor=south,
        },]
            \addplot [domain=-3:3, smooth, thick] { 6*x^2 + x^4 - 6*x };
        \end{axis}
     
    \end{tikzpicture}
    
\end{minipage}
\hfill
\begin{minipage}{0.45\textwidth}
    \begin{tikzpicture}
        \begin{axis} [axis lines=center, xtick=\empty, ytick=\empty, xlabel=$\phi$, ylabel=$V(\phi)$,
        every axis x label/.style={
            at={(ticklabel* cs:1.0)},
            anchor=west,
        },
        every axis y label/.style={
            at={(ticklabel* cs:1.0)},
            anchor=south,
        },]
            \addplot [domain=-3:3, smooth, thick] { -6*x^2 + x^4 - 2*x };
        \end{axis}
     
    \end{tikzpicture}
\end{minipage}
\label{fig:breaking_O1_symmetry}
\caption{The introduction of the boson-fermion interaction, with a finite fermionic mass, causes the breaking of the O(1) symmetry. It shifts the equilibrium position in the symmetric phase (left) causing $\left\langle \phi \right\rangle = 0$, and tilts the potential in the broken phase (right), making the two minima not equivalent.}
\end{figure}


\section{Dynamics}
\textcolor{red}{This has to be rewritten for Wilson fermions, in a discretised way}
The effective action after solving the fermionic path integral reads
\begin{equation*}
    S[\phi] = S_\phi - \underset{x,s,f}{\tr}log{D(\phi)} = \int_x \left[ \phi_x \, \left(-\frac{\partial_x^2}{2} + \frac{m_\phi^2}{2}\right) \, \phi_x + \frac{\lambda}{4!} \, \phi_x^4 \right] - \underset{x,s,f}\tr\log{D(\phi_x)}
\end{equation*}
with $D(\phi) = \slashed{\partial} + m_q + g\phi$. \\
One can derive the classical equations of motions by imposing $\frac{\delta S}{\delta \phi} = 0$. \\ For $\lambda = 0$ they assume a simple form
\begin{equation*}
    \phi(x) = \frac{g}{m_\phi^2 + k^2} \ \underset{s,f}{\tr} \left[D^{-1}(\phi(x)\right]
\end{equation*}
In the discretised theory
\begin{equation*}
    S[\phi] = \dots
\end{equation*}
one can calculate the drift for the Langevin evolution as
\begin{equation*}
    \frac{\delta S}{\delta \phi(x)} = \frac{\delta S_\phi}{\delta \phi(x)} - \tr \left[D^{-1} \frac{\partial D}{\partial \phi(x)}\right] = \frac{\delta S_\phi}{\delta \phi(x)} - g \, \tr \left[D^{-1}\right]
\end{equation*}
where $D$ is the Wilson-Dirac operator given by equation \eqref{eq:wilson-dirac_operator}. \\
To evaluate the trace we use the bilinear noise scheme \textcolor{red}{add reference} which is illustrated in Appendix \ref{AppendixC}.
The full quantum dynamics is described by \eqref{eq:Langevin_scalar_full}. One can find an analytical solution, in some limiting cases. For example let us consider a classical simulation with $\lambda = 0, D = m_q + g\phi$. Equilibrium will be reached when $\frac{\delta S}{\delta \phi(x)} = 0$ which means, assuming zero momentum,
\begin{equation*}
    m_\phi^2 \, \phi(x) = g \tr \left[D^{-1}(\phi(x))\right]
\end{equation*}
Since the fermionic kinetic term has been removed from the action by assumption and no fluctuations, except from numerical issues, are present in the simulation, the Dirac operator is diagonal and can be inverted explicitly
\begin{equation}
    D^{-1}(\phi(x)) = \frac{1}{m_q + g \, \phi(x)}
    \label{eq:classical_eq_motion_approx}
\end{equation}
yielding $\tr D^{-1} =  4 \left(m_q + g \, \phi\right)^{-1}$. \\
Inserting this result into \eqref{eq:classical_eq_motion_approx} and solving for $\phi(x)$ one finds the two field configurations that minimise the action, in accordance to what pictured in figure \ref{fig:breaking_O1_symmetry}

\begin{equation*}
    \phi^*_{1,2} = \frac{m_q}{2 \, g} \left(-1 \pm \sqrt{1 + 16 \, g^2 / m_q^2 m_\phi^2}\right)
\end{equation*}

\newpage 



\section{Relevant quantities}
Two-points function
\begin{align*}
    \left\langle \psi(x) \, \bar\psi(y) \right\rangle 
    &= \frac{1}{Z} \, \int \mathcal{D}\phi \, \mathcal{D}\psi \, \mathcal{D}\bar\psi \ \psi(x) \, \bar\psi(y) \, \exp \left( - S_\phi - \psi D \psi + \bar\eta \psi + \bar \psi \eta \right) \\
    &= \frac{1}{Z} \, \int \mathcal{D}\phi \, \mathcal{D}\psi \, \mathcal{D}\bar\psi \ \frac{\delta}{\delta \bar \eta(x)} \frac{\delta}{\delta \eta(y)} \, \exp \left( - S_\phi - \psi D \psi + \bar\eta \psi + \bar \psi \eta \right) \\
    &= \frac{1}{Z} \, \int \mathcal{D}\phi \ \text{det}\left[D(\phi)\right] \ \exp \left( - S_\phi \right) \ \frac{\delta}{\delta \bar \eta(x)} \frac{\delta}{\delta \eta(y)} \, \exp\left( \bar\eta D^{-1} \eta \right) \\
    &= \left\langle \left[D^{-1}(\phi)\right](x,y)\right\rangle
\end{align*}

On the lattice it still holds that 
\begin{equation*}
     \left\langle \psi_m \, \bar\psi_n \right\rangle =     \left\langle \left[D^{-1}(\phi)\right]_{mn}\right\rangle
\end{equation*}
with $D$ beeing the Wilson Dirac operator REFERENCE

Now consider a source at the origin at time $t=0$, namely a state $\psi(t,x) = c \, \delta_{t,0} \, \delta_{x,0}$. After letting it propagate for a time $t$, me might ask how is a point on the lattice, on average, correlated to the initial source. This is quantified by the \emph{correlator}, which, on the lattice, it is defined as
\begin{align*}
    C(n_t,0) \equiv \frac{1}{N_x} \sum_{n_x} \left[\left\langle \psi(n_t, n_x) \, \bar\psi(0,0)\right\rangle + \left\langle \psi(N_t-n_t, n_x) \, \bar\psi(0,0) \right] \right\rangle
\end{align*}
Note that we sum up two waves because the source propagates both forward and backward in time due to the boundary conditions. \\
Since for $t \to \infty$ one has that $C(t,p) \propto e^{-E_0(p) t}$, we expect 
\begin{equation*}
    C(t,p) \approx \sinh \left(E_0 \left(\frac{N_t}{2} - t\right)\right)
\end{equation*}

Pole mass, renormalized mass, effective mass, bare mass, physical mass


In momentum space
\begin{equation*}
    \tilde D(p,q) = m + g \, \sigma + i \, g \, \gamma^5 \, \tau^j \pi^j + \sum_{\mu} \sin^2\left(\frac{p_\mu}{2}\right) + \sum_\mu \gamma_\mu \sin\left(p_\mu\right)
\end{equation*}
Or making explicit the flavour and spinor components
\begin{gather*}
     \tilde D(p,q) = M \, \mathds{1}_f \otimes \mathds{1}_s  + i \, g \, \pi^j \, \tau^j \otimes \gamma^5   + \left(\sum_\mu \sin\left(p_\mu\right) \mathds{1}_f \otimes \gamma_\mu\right)
\end{gather*}
where 
\begin{equation*}
   M = m + g \, \sigma + \sum_{\mu} \sin^2\left(\frac{p_\mu}{2}\right)
\end{equation*}









Effective action
\begin{equation*}
    S_{\text{eff}} = S_\phi + \text{Tr} \log D(\phi)
\end{equation*}
Drift force
\begin{equation*}
    K_{\phi^j} = - \frac{\delta S}{\delta \phi^j} = - \frac{\delta S_\phi}{\delta \phi^j} - \text{Tr} \left[ D^{-1} \, \frac{\delta D}{\delta \phi^j} \right]
\end{equation*}

\begin{align*}
    \frac{\delta D}{\delta \phi^j} = 
    \begin{cases}
        g \qquad &\text{if } j = 0 \\
        i \, g \, \gamma^5 \, \tau^j \qquad &\text{if } j = 1,2,3
    \end{cases}
\end{align*}

Bilinear noise scheme

\begin{equation*}
    \text{Tr} \left[ D^{-1} \, \frac{\delta D}{\delta \phi^j} \right] \approx \left\langle \eta \right|  D^{-1} \, \frac{\delta D}{\delta \phi^j} \left| \eta \right\rangle = 
    \left\langle \psi \right| \frac{\delta D}{\delta \phi^j} \left| \eta \right\rangle
    \qquad\qquad \left| \psi \right\rangle = D^{-1} \left| \eta \right\rangle = D^\dagger \, \underbrace{(D D^\dagger)^{-1} \left| \eta \right\rangle}_{\text{CG}}
\end{equation*}

\begin{align*}
    \tau^1 \otimes \gamma^5 &= 
    \begin{pmatrix*}
        0 & 1 \\
        1 & 0
    \end{pmatrix*}
    \otimes
    \begin{pmatrix*}
        0 & i \\
        -i & 0
    \end{pmatrix*}
    & = &
    \begin{pmatrix*}
        0 & 0 & 0 & i \\
        0 & 0 & -i & 0 \\
        0 & i & 0 & 0 \\
        -i & 0 & 0 & 0
    \end{pmatrix*}
    \\~\\
     \tau^2 \otimes \gamma^5 &= 
    \begin{pmatrix*}
        0 & -i \\
        i & 0
    \end{pmatrix*}
    \otimes
    \begin{pmatrix*}
        0 & i \\
        -i & 0
    \end{pmatrix*}
    & = &
    \begin{pmatrix*}
        0 & 0 & 0 & 1 \\
        0 & 0 & -1 & 0 \\
        0 & -1 & 0 & 0 \\
        1 & 0 & 0 & 0
    \end{pmatrix*}
    \\~\\
     \tau^2 \otimes \gamma^5 &= 
     \begin{pmatrix*}
        1 & 0 \\
        0 & -1
    \end{pmatrix*}
    \otimes
    \begin{pmatrix*}
        0 & i \\
        -i & 0
    \end{pmatrix*}
    & = &
    \begin{pmatrix*}
        0 & i & 0 & 0 \\
        -i & 0 & 0 & 0 \\
        0 & 0 & 0 & -i \\
        0 & 0 & i & 0
    \end{pmatrix*}
\end{align*}







Quark-meson model action in the continuum
\begin{align*}
    S[\sigma, \pi^j, \psi, \bar\psi] & = \int_x \sum_{j=1}^3 \{ \frac{1}{2} \ \left(\partial_\mu \sigma\right)^2 + \frac{1}{2} \left(\partial_\mu \pi^j \right)^2 + \frac{m^2}{2} \left(\sigma^2 + (\pi^j)^2\right) + \frac{\lambda}{4!} \left(\sigma^2 + (\pi^j)^2\right)^2 \\ 
    & + \bar \psi \left( \slashed{\partial} + m_q \right) \psi + \bar \psi \left(g \, \sigma + i \, g \,\gamma^5 \pi^j \tau^j\right) \psi \}
\end{align*}





