\chapter{Theoretical background}
\label{chap:theoretical_background}

\section{Lattice formulation of Euclidean Quantum Field Theory}
\label{sec:lattice_formulation}
In the euclidean formulation of quantum field theory, one typically defines the path integral as 
\begin{equation}
    Z = \int \mathcal{D}\phi\mathcal{D}\psi\mathcal{D}\psi \ e^{-S[\phi, \psi, \bar\psi]}
    \label{eq:path_integral_generic}
\end{equation}
and aims to compute correlation functions as
\begin{equation*}
        \left\langle \xi_{x_1} \dots \xi_{x_n}  \right\rangle = \frac{1}{Z} \int \mathcal{D}\phi\mathcal{D}\psi\mathcal{D}\psi \ \xi_{x_1} \dots \xi_{x_n} \ e^{-S[\phi, \psi, \bar\psi]} \qquad \xi_{x_i} \in \{\phi_{x_i}, \psi_{x_i}, \bar\psi_{x_i}\} \\
\end{equation*}

\section{Yukawa theory}
\label{sec:Yukawa_theory}
Let us consider the Yukawa theory defined by the action
\begin{equation}
\begin{aligned}
    S[\phi,\psi,\bar\psi] &= S_\phi[\phi] + S_\psi[\psi, \bar\psi] + S_\text{int}[\phi, \psi, \bar\psi]\\
     S_\phi[\phi] &= \int_x \phi_x \left(-\frac{\partial^2_x}{2} + \frac{m_\phi^2}{2}\right) \phi_x + \frac{\lambda}{4!} \, \phi_x^4 \\
     S_\psi[\phi, \psi, \bar\psi] &= \int_x \bar \psi_x \left(\slashed{\partial}_x + m_q \right) \psi_x \\
     S_\text{int}[\phi, \psi, \bar\psi] &= g \, \int_x \bar \psi_x \, \phi_x \, \psi_x
    \label{eq:yukawa_continuum}
\end{aligned}
\end{equation}
One can see that the action is made of a scalar part $S_\phi[\phi]$, a fermionic part $S_\psi[\psi, \bar\psi]$ and a Yukawa interaction term $S_\text{int}[\phi, \psi, \bar\psi]$. \\
It is also convenient for later purposes to define the operators $K, D$ as 
\begin{equation}
    \begin{aligned}
        K &=  -\frac{\partial^2_x}{2} + \frac{m_\phi^2}{2} \\
        D &= \slashed{\partial} + m_q + g\phi 
    \end{aligned}
    \label{eq:definition_kinetic_terms_continuum}
\end{equation}
This allows one to rewrite the action as
\begin{equation*}
    S[\phi,\psi,\bar\psi] = \int_x \phi_x K \phi_x + \frac{\lambda}{4!}\phi_x^4 + \bar\psi_x D \psi_x
\end{equation*}
For vanishing quark mass the action is fully invariant under the chiral transformation
\begin{equation}
    \begin{aligned}
        \phi &\to -\phi \\
        \psi &\to e^{i\alpha} \, \psi \\
        \bar\psi &\to \bar\psi \, e^{-i\alpha} \\
    \end{aligned}
    \label{eq:yukawa_continuum_rewritten}
\end{equation}
The main feature of the model is chiral symmetry breaking \cite{Nambu1961DynamicalI, Nambu1961DynamicalII}, which can happen explicitly at the level of the classical action for a non-zero quark mass, or spontaneously when the scalar field gains a non-zero expectation value. One can in fact notice already by looking at \eqref{eq:definition_kinetic_terms_continuum}, that $\left\langle\phi\right\rangle \neq 0$ has the same effect on the action as a finite bare quark mass. This observation will be made more quantitative in section (SECCCC) where it will be shown that  
\begin{equation*}
    \left\langle \phi \right\rangle \sim \left\langle \bar \psi \psi \right\rangle \sim D^{-1}
\end{equation*}
The fermionic part of the path integral \eqref{eq:path_integral_generic} can be performed explicitly
\begin{equation*}
    \int \mathcal{D} \bar\psi \mathcal{D} \psi \ \exp\left( - \int_x \bar\psi_x \,  D \, \psi_x \right) = \text{det} \, D[\phi] = e^{\tr \log (D[\phi])}
\end{equation*}
where the trace is performed over space-time, flavour and spinor components. \\ 
The full path integral can now be expressed in terms of the resulting effective action for the scalar fields
\begin{equation*}
    Z = \int \mathcal{D}\phi \ e^{-S_\text{eff}[phi]}
\end{equation*}
with
\begin{equation*}
    S_\text{eff}[\phi] = S_{\phi}[\phi] - \underset{x,s,f}{\tr} \log D[\phi]
\end{equation*}
One can derive the classical equations of motion by imposing $\frac{\delta S}{\delta \phi} = 0$, here expressed in momentum space
\begin{equation*}
     (k^2 + m_\phi^2) \, \phi(x) + \frac{\lambda}{6} \, \phi^3(x) = g \ \underset{s,f}{\tr} \left[D^{-1}(\phi(x)\right] = - g \ \bar\psi(x) \psi(x)
\end{equation*}
where the trace is performed over spin and flavour components. For $\lambda = 0$, they highlight a simple proportionality relation between magnetisation and chiral condensate, which for zero momentum reads
\begin{equation}
    \phi(x) = - \frac{g}{m_\phi^2} \ \bar \psi(x) \psi(x)
    \label{eq:classical_EOM}
\end{equation}
The classical relation \eqref{eq:classical_EOM} is proven to hold also at mean field on the quantum level \cite{Buballa2005NJL-modelMatter} and will be studied in the discretised theory in section \ref{sec:classical_to_quantum}. \\~\\


\begin{figure}[h]
\centering
\begin{minipage}{0.45\textwidth}
    \begin{tikzpicture}
        \begin{axis} [axis lines=center, xtick=\empty, ytick=\empty, xlabel=$\phi$, ylabel=$V(\phi)$,
        every axis x label/.style={
            at={(ticklabel* cs:1.0)},
            anchor=west,
        },
        every axis y label/.style={
            at={(ticklabel* cs:1.0)},
            anchor=south,
        },]
            \addplot [domain=-3:3, smooth, thick] { 6*x^2 + x^4 - 6*x };
        \end{axis}
     
    \end{tikzpicture}
    
\end{minipage}
\hfill
\begin{minipage}{0.45\textwidth}
    \begin{tikzpicture}
        \begin{axis} [axis lines=center, xtick=\empty, ytick=\empty, xlabel=$\phi$, ylabel=$V(\phi)$,
        every axis x label/.style={
            at={(ticklabel* cs:1.0)},
            anchor=west,
        },
        every axis y label/.style={
            at={(ticklabel* cs:1.0)},
            anchor=south,
        },]
            \addplot [domain=-3:3, smooth, thick] { -6*x^2 + x^4 - 2*x };
        \end{axis}
     
    \end{tikzpicture}
\end{minipage}
\label{fig:breaking_O1_symmetry}
\caption{The introduction of the boson-fermion interaction, with a finite fermionic mass, causes the breaking of the O(1) symmetry. It shifts the equilibrium position in the symmetric phase (left) causing $\left\langle \phi \right\rangle = 0$, and tilts the potential in the broken phase (right), making the two minima not equivalent.}
\end{figure}

\newpage

\section{Discrete formulation}
\label{sec:lattice_discretisation}
In this section we provide a discretised formulation of the Yukawa model introduced in section \ref{subsec:Yukawa_theory}. \\~\\
For what concerns the bosonic part of the action, a discretisation can be done straightforwardly with the following replacements
\begin{equation*}
    \begin{aligned}
        \int_x \qquad &\to \qquad a^2 \sum_x \\
        \partial^2_x = \frac{\partial^2}{\partial t^2} + \frac{\partial^2}{\partial x_1^2} \qquad &\to \qquad \sum_\mu \left[\frac{\delta_{m,n+\mu} + \delta_{m,n-\mu} - 2 \delta_{m,n}}{a^2}\right]
    \end{aligned}
\end{equation*}
which yields to the lattice action
\begin{equation*}
        S_\phi [\phi] =  a^2 \sum_{m,n} \phi_m \, K_{mn} \, \phi_n + \frac{\lambda}{4!} \, \sum_n \phi_n^4 
\end{equation*}
with 
\begin{equation*}
    K_{mn} = - \sum_\mu \left[\frac{\delta_{m,n+\mu} + \delta_{m,n-\mu} - 2 \delta_{m,n}}{a^2}\right] + m_\phi^2 \delta_{mn} 
\end{equation*}
One can also express everything using dimensionless couplings
\begin{equation}
    \begin{aligned}
        \hat m_\phi^2 &= a^2 \, m_\phi^2 \\
        \hat \lambda &= a^{2} \, \lambda, \\
        \hat K_{mn} &= a^2 K_{mn}
    \end{aligned}
    \label{eq:couplings_redefitinion}
\end{equation}
and the action is then described only in terms of dimensionless quantities
\begin{equation*}
    S_\phi=-\sum_{n, \mu} \hat\phi_n \hat\phi_{n+\mu}+\sum_n\left[\frac{1}{2}\left(4+\hat m^2\right) \hat\phi_n^2 +\frac{\hat\lambda}{4 !} \, \hat\phi_n^4\right]
\end{equation*}
\textcolor{red}{Otherwise it is customary to introduce dimensionless couplings $\kappa, \beta$ defined via
\begin{equation}
    \begin{aligned}
       \phi & \rightarrow(2 k)^{\frac{1}{2}} \phi, \\
        (a m)^2 & \rightarrow \frac{1-2 \beta}{k}-4, \\
        a^{-2} \lambda & \rightarrow \frac{6 \beta}{k^2}
    \end{aligned}
    \label{eq:dimless_couplings_defitinion}
\end{equation}
and the equivalent action reads
\begin{equation*}
    S_{\phi}=-2 k \sum_{n, \mu} \phi_n^i \phi_{n+\mu}^i+(1-2 \beta) \sum_n \phi_n^i \phi_n^i+\beta \sum_n\left(\phi_n^i \phi_n^i\right)^2
\end{equation*}
In the following, we might use any of the two dimensionless formulations interchangeably, since they are completely equivalent given the definitions \eqref{eq:couplings_redefitinion}, \eqref{eq:dimless_couplings_defitinion}.} \\
For what concerns the fermionic action, a naive discretisation is not sufficient, due to the well known doubling problem. In this work Wilson fermions are employed as a way to fix such issue. Details of this formulation are explained in section \ref{AppendixB}. Here, only the final discretised action is reported, which reads
\begin{equation*}
    S_\psi[\phi, \psi, \bar\psi] = \sum_{m,n} \bar\psi_m \, D_{m,n} \psi_n + g \, \sum_n \bar\psi_n \phi_n \psi_n
\end{equation*}
with $\psi_n$ beeing a four-component spinor (2 flavour components and 2 Dirac components), and $D_{m,n}$ beeing the Wilson-Dirac operator \note{is $g\phi$ included in the definition of $D$?)} defined as 
\begin{equation}
    \begin{aligned}
    D_{m, n} = &- \left(\frac{\Gamma_{+0}}{2} \, \delta_{m, m+0} +\frac{\Gamma_{-0}}{2} \, \delta_{m, m-0} + \frac{\Gamma_{+1}}{2} \, \delta_{m, m+1} + \frac{\Gamma_{-1}}{2} \, \delta_{m, m-1}\right) \, \delta _{f, f'} \\
     &+ \left(2 + m + g\phi\right) \ \delta_{s,s'} \delta_{m,n} \\
    \end{aligned}
    \label{eq:wilson-dirac_operator}
\end{equation}
The Wilson projectors $\Gamma_{\pm \mu}$ are defined as
\begin{equation*}
    \Gamma_{\pm \mu} = 1 \mp \gamma_\mu 
\end{equation*}
One can then proceed by defining dimensionless fields and couplings
\begin{equation*}
    \begin{aligned}
        \hat\psi &= \rightarrow a^{\frac{1}{2}} \psi, \\
        \hat m_q &= a m_q, \\
        \hat g &= a g
    \end{aligned}
    \label{eq:fermionic_theory_dimless_redefinitions}
\end{equation*}
to describe the action only in terms of dimensionless quantities. \\
In the remaining of this work, the dimensionless couplings and fields will be adopted, unless otherwise specified. Additionally, both the original action $S$ and the effective action $S_\text{eff}$ will be denoted by $S$ for simplicity. It will be clear from the context to which of the two we will be referring. \\


\section{Block spin RG}
\label{sec:blockspin}
Cite Kadanoff article \cite{PhysicsPhysiqueFizika.2.263}. Average spins and rescale stuff to keep correlation length fixed.

\section{Wilson RG}
\label{sec:wilson_rg}
Extends block spin RG. \\
One splits the fields as $\Phi = \phi + \varphi$ where $\phi$ are fields with momenta $p \leq b\Lambda$ and $\varphi$ are fields with momenta $b\Lambda < p < \Lambda$, then one writes the path integral in terms of the Wilsonian effective action
\begin{equation*}
    Z = \int D\Phi_\Lambda \, e^{-S_\Lambda[\Phi]} = \int D\phi_{b\Lambda} \, e^{-S_{b\Lambda}[\phi]} \int D\varphi_{b\Lambda, \Lambda}  \, e^{-S_{b\Lambda, \Lambda}[\phi, \varphi]} = \int D\phi_{b\Lambda} \, e^{-S_{b\Lambda}^\text{eff}[\phi]}
\end{equation*}
where 
\begin{equation*}
    S_{b\Lambda}^\text{eff}[\phi] = S_{b\Lambda}[\phi] - \log\left( \int D\varphi_{b\Lambda, \Lambda}  \, e^{-S_{b\Lambda, \Lambda}[\phi, \varphi]}\right) =  S_{b\Lambda}[\phi] + \Delta S_{b\Lambda, \Lambda}[\phi]
\end{equation*}
Note that all the steps above are exact identities. In particular, performing the integral over ultraviolet modes, is the continuum version of the block spinning procedure outlined in the previous section. \\
Note also that $S_{b\Lambda}[\phi]$ is the same as the initial action, but it is non-zero only for fields with $p^2 \leq b\Lambda^2$. \\
At this point one can expand $\Delta S_{b\Lambda, \Lambda}$ in powers of the field \textcolor{red}{(before, another step, see jan pawlowski's notes)}. Powers that are present also in $S_{b\Lambda}$ can be absorbed into the latter by redefining the coupling
\begin{equation}
\begin{aligned}
& \phi^{\prime}\left(x^{\prime}\right)=\left[b^{2-d}(1+\Delta z)\right]^{\frac{1}{2}} \phi(x), \quad m^{\prime 2}=\left(m^2+\Delta m^2\right) \frac{1}{1+\Delta z} \frac{1}{b^2}, \quad \lambda^{\prime}=(\lambda+\Delta \lambda) \frac{1}{(1+\Delta z)^2} b^{d-4}, \\
& \alpha^{\prime}=(\alpha+\Delta \alpha) \frac{1}{(1+\Delta z)^2} b^d, \quad \lambda_6^{\prime}=\left(\lambda_6+\Delta \lambda_6\right) \frac{1}{(1+\Delta z)^3} b^{2 d-6}, \quad \ldots \\
&
\end{aligned}
\end{equation}
higher powers are suppressed (non-renormalisable terms) are suppressed. By neglecting these higher order powers, one can bring the action in the same form as the initial one via redefinition of the parameters. The whole procedure is non-perturbative. The result can be compared to the initial action after redefinition of all the dimensionful quantities i.e. via $p'\equiv p/b$. 

\section{Continuum limit}
\label{sec:continuum_limit}
Introduce renormalisation as a mapping as in page 40 of Montvay Munster.
We are interest in the set of theories in theories space that have constant renormalised couplings (trajectories in Kadanoff-Wilson RG) but different dimless masses $m_{q,r} \, a$. Since the latter gives the cutoff, we are moving the cutoff while keeping the other renormalised couplings fixed. How does this happen? We have to compensate the change in the cutoff by a change in the bare couplings as detailed here. IN perturbation theory we can compute renormalised couplings as functions of the bare ones. We can the (HOPEFULLY) invert these relations and determine bare parameters as functions of the renormalised ones. Near the continuum limit we can expect renormalised vertex functions to depend only weakly on the spacing so that we can impose
\begin{equation*}
    a \frac{d}{da} \Gamma_R^n(p^i, g^i_R, m_{q,r} \, a) = 0
\end{equation*}
And since 
\begin{equation*}
    \Gamma_R = Z \Gamma_0
\end{equation*}
one gets
\begin{equation*}
    \left(a\partial_a - \beta_\text{lat} \partial g_0 + n \gamma_\text{lat}\right) \ \Gamma_0^n = 0
\end{equation*}
Integrating these equations leaves us with functions
\begin{equation*}
    g_0^i(a)
\end{equation*}
that tell us how to change bare couplings as a back reaction to a change in the spacing in order to keep physics constant. \\
In this perspective we want renormalised fixed and we change bare ones. 