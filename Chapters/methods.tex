\chapter{Methods and algorithms}
\section{Langevin equation}
In order to compute expectation values from the discretised path integral \textcolor{red}{add ref.} we employ a Langevin Monte Carlo algorithm, which is based on stochastic quantisation \cite{ParisiWu, Damgaard1987StochasticQuantization}. \\
The idea is that Euclidean Quantum Field theory can be thought as a system in thermal equilibrium with a heat reservoir and hence described as a stochastic process via the Langevin equation. \\
Let us consider a scalar field $\phi$ with a Euclidean action $S[\phi]$ and the following Langevin equation
\begin{equation}
    \partial_\tau \phi(\tau, x) = - \frac{\delta S[\phi]}{\delta \phi (\tau, x)} + \eta (\tau, x)
    \label{eq:Langevin_scalar_full}
\end{equation}
where $- \frac{\delta S[\phi]}{\delta \phi (\tau, x)}$ is the drift term and $\eta (\tau, x)$ is a random white noise field defined by
\begin{equation*}
    \expect{\eta(x,\tau)} = 0 \qquad \expect{\eta(x,\tau) \, \eta(x',\tau')} = 2 \, \delta(x, x') \, \delta (\tau, \tau')
\end{equation*}
For $t \to +\infty$ \textcolor{red}{(assuming a stationary solutions exists, but I think it is always the case if the action is bounded from below)} one can prove that \textcolor{red}{ADD CITATION} the stationary probablity distribution is given by
\begin{equation}
    \mathcal{P}(\phi) = \frac{1}{Z} \, \exp\left(-S[\phi])\right)
    \label{eq:probability_field_configuration_full}
\end{equation}
This allow one to compute correlation functions as moments of the distribution \eqref{eq:probability_field_configuration_full}. \\
Equation \ref{eq:Langevin_scalar_full} can be integrated numerically for discrete time steps $\tau_n$ via, for example, an explicti Euler-\textcolor{red}{Someone Else} scheme
\begin{equation*}
    \phi(\tau_{n+1}, x) = \phi(\tau_{n}, x) - \epsilon \,  \frac{\delta S[\phi]}{\delta \phi (\tau_n, x)} + \sqrt{2\epsilon} \, \eta(\tau_n, x)
\end{equation*}
Higher order integration schemes are, for example, \textcolor{red}{cite schemes}.
For the discretised action of the Yukawa theory the drift reads
\begin{equation*}
    \begin{aligned}
        \frac{\partial S}{\partial \phi(\tau_n,m)} &= \frac{\partial S_\phi}{\partial \phi(\tau_n, m)} - \underset{s,f}{\tr} \left[D^{-1} \frac{\partial D(\phi)}{\partial \phi(\tau_n, m)}\right] \\
        &= \frac{\partial S_\phi}{\partial \phi(\tau_n, m)} - g \, \underset{s,f}{\tr} \left[D^{-1}(\phi(\tau_n,m))\right]
    \end{aligned}
\end{equation*}
To evaluate the trace, which is due to the fermionic contribution, we use the bilinear noise scheme \textcolor{red}{add reference} which is illustrated in Appendix \ref{AppendixC}.



\section{Coloured noise}
\label{sec:coloured_noise}
\note{Mention some possible uses of colored noise beside taking continuum limits.} \\
Connection to stochastic regularisation \cite{}
In the stochastic quantisation procedure the noise which accounts for the quantum fluctuations of the theory is assumed to be white noise. This means that its power spectrum is flat in momentum space, extending in all the first Brillouin zone, namely for $p_\mu \in [-\pi/a, \pi/a]$. One could modify this spectrum, and in this case one says \emph{colored noise}. In particular one could put a sharp cutoff on the total momentum, imposing $p^2 \leq \Lambda^2$. We refer to this particular case as \emph{regularised noise}. \\
In such case the Langevin equation for the scalar field \eqref{eq:Langevin_scalar_full} assumes the form
\begin{equation*}
    \partial_\tau \phi(\tau, x) = - \frac{\delta S[\phi]}{\delta \phi (\tau, x)} + r_\Lambda (x) \, \eta (\tau, x)
    \label{eq:Langevin_scalar_regularised}
\end{equation*}
where the regularising function $r_\Lambda(x)$ can be easilly expressed in momentum space as $r_\Lambda(p) = \theta(\Lambda^2 - p^2)$. One can show \cite{Pawlowski2017CoolingNoise} that the stochastic process is now driven towards a new equilibrium distribution
\begin{equation}
    \mathcal{P}_\Lambda(\phi) = \frac{1}{Z} \, \exp\left(-S_\Lambda[\phi])\right) = \frac{1}{Z} \, \exp\left(-(S[\phi] + \Delta S_\Lambda[\phi])\right)
    \label{eq:probability_field_configuration_regularised}
\end{equation}
where the correction term $S_\Lambda[\phi]$ ensures that the probability measure $\mathcal{P}_\Lambda$ vanishes for squared fields' momenta greater than the cutoff $\Lambda^2$. \\
An explicit example of such regulator for a free scalar field can be \textcolor{red}{ADD EQUATION PAWLOWSKI}.
This is just one example of regularisation and different ones can be chosen. See \cite{Pawlowski2017CoolingNoise} for details.


\section{Lattice QFT with regularised noise}
After the general introduction on coloured noise given in the previous paragraph, let us now look more closely on the lattice formulation and at the various applications of such techniques.
From a code perspective, the algorithm to regularise noise with a sharp cutoff is presented in Appendix WHICH ONE?. \\
Let us consider a squared two-dimensional lattice with size $L \equiv L_x = L_y$ and spacing $a \equiv a_x = a_y$. This implies a maximum momentum $p_\text{max} = \pi / a$ in each space-time direction. We consider a regularised simulation with cutoff $\Lambda'$. Let us also define $\Lambda^2 \equiv (p^x_\text{max})^2 + (p^y_\text{max})^2$. At this point we introduce a parameter, called \emph{cutoff fraction}, defined as $s^2 \equiv \Lambda^2 / \Lambda'^{\,2}$. With this notation, $s=1$ is the full white noise case, while for any regularised noise one has $0 < s < 1$. \\
We now want to address the following question: given the stationary probability distribution \ref{eq:probability_field_configuration_regularised}, is it possible to compensate the change in physical observables caused by the removal of the quantum modes via regularised noise, by a rescaling of the bare parameters that enter the lattice discretised action? Remembering from sections \ref{sec:lattice_discretisation} and \ref{sec:continuum_limit} that $\Lambda \sim a^{-1}$, this question would also address the problem of continuum limit of effective field theories. In fact the question we want to address is completely equivalent to the following: can one compensate a change in the spacing (controlled by the bare parameters), by a change of the noise in the simulation, in order to keep physical observables constant?\\
To this end, let us consider a noise regularisation given by $s^2 = \Lambda^2 / \Lambda'^{\,2} < 1$ ($s = a' / a < 1$), which means a higher cutoff (a smaller spacing $a'$). We now introduce an approximate ansatz which is based on the analogy to standard block-spin transformations. Higher order corrections to this simple ansatz are discussed in WHICH SECTION? \\
A change in the spacing $a \to ra$ will cause the following
One then accomodates the change of the spacing (cutoff) by a change in the bare dimensionful couplings, as explained in chapter \ref{chap:theoretical_background}. At lowest order one can perform standard block spin step, which corresponds to a tree level analysis in the wilsonian perspective. This means that all the dimensionful quantities, couplings, momenta, fields, have to be rescaled according to their dimension, as detailed in the following lines. \\
For what concerns the scalr part of the action, the rescaling at tree level is rather trivial 
\begin{equation*}
    m_\phi^2 \to s^2m_{\phi}^2, \quad \lambda \to s^s\lambda, \quad \phi \to \phi \quad
\end{equation*}
The fermionic parts need a more careful analysis of the drift term, since the fermionic fields are not explictly present in the action (REF TO EQ.) due to their Grassmman nature. \\
We first recall the relation
\begin{equation*}
    \bar\psi \psi = a^{-2} \, \underset{s,f}{\tr} \, D^{-1}
\end{equation*}
which we rewrite as 
\begin{equation*}
    \Sigma = a^{-2} \, \chi
\end{equation*}
for simplicity. \\
The drift term is
\begin{equation*}
    K_n = - \frac{\partial S_{latt}}{\partial \phi_n} = K_{\phi} - g \bar\psi_n\psi_n
\end{equation*}
To compute the fermionic contribution we would like to use the relation (SIGMA, CHI). Thus, one has to take into account that the rescaling of the fermionic fields, which are dimensionful, cause a rescaling in the fermionic contribu
\begin{table}[]
    \centering
    \begin{tabular}{cccccccc}
        \toprule
         $s$ & $a$ & $\Lambda$ & $N$ & $m_\phi^2$ & $\lambda$ & $m_{q, 0}$ & $g_0$  \\
         \midrule 
         $1$ & $a$ & $\Lambda$ & $8$ & $m_{\phi, 0}^2$ & $\lambda$ & $m_{q, 0}$ & $g_0$ \\
         $2$ & $a/2$ & $2\Lambda$ & $16$ & $m_{\phi, 0}^2 / 4$ & $\lambda / 4$ & $m_{q, 0}$ & $g_0$ \\
         $4$ & $a/4$ & $4\Lambda$ & $32$ & $m_{\phi, 0}^2 / 16$ & $\lambda / 16$ & $m_{q, 0}$ & $g_0$ \\
         $8$ & $a/8$ & $8\Lambda$ & $64$ & $m_{\phi, 0}^2 / 64$ & $\lambda / 64$ & $m_{q, 0}$ & $g_0$ \\
         \bottomrule
    \end{tabular}
    \caption{Rescaling}
    \label{block_spin_steps}
\end{table}


\newpage 






