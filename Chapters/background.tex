\chapter{Theoretical background}

\section{Yukawa theory}
\label{subsec:Yukawa_theory}
Let us consider the Yukawa theory defined by the action
\begin{equation}
\begin{aligned}
    S[\phi,\psi,\bar\psi] &= S_\phi[\phi] + S_\psi[\phi, \psi, \bar\psi] \\
     S_\phi[\phi] &= \int_x \phi_x \left(-\frac{\partial^2_x}{2} + \frac{m_\phi^2}{2}\right) \phi_x + \frac{\lambda}{4!} \, \phi_x^4 \\
     S_\psi[\phi, \psi, \bar\psi] &= \int_x \bar \psi_x \left(\slashed{\partial}_x + m_q + g \, \phi_x \right) \psi_x
    \label{eq:yukaw_continuum}
\end{aligned}
\end{equation}
\section{QM model}
Slightly more technical: what is QM model and why it is useful for QCD
\section{Lattice discretisation}
\label{sec:lattice_discretisation}
\textbf{NOTE} We start from discretising the scalar theory O(N). We then introduce Wilson fermions (with details in appendix) and we write down the discretised version of the action for NJL and QM. \\~\\
In this section we provide a discretised formulation of the Yukawa model introduced in section \ref{subsec:Yukawa_theory}. \\~\\
For what concerns the bosonic part of the action, a discretisation can be done straightforwardly with the following replacements
\begin{equation*}
    \begin{aligned}
        \int_x \qquad &\to \qquad a^2 \sum_x \\
        \partial^2_x = \frac{\partial^2}{\partial t^2} + \frac{\partial^2}{\partial x_1^2} \qquad &\to \qquad \sum_\mu \left[\frac{\delta_{m,n+\mu} + \delta_{m,n-\mu} - 2 \delta_{m,n}}{a^2}\right]
    \end{aligned}
\end{equation*}
which yields to the lattice action
\begin{equation*}
        S_\phi [\phi] =  a^2 \sum_{m,n} \phi_m \, K_{mn} \, \phi_n + \frac{\lambda}{4!} \, \sum_n \phi_n^4 
\end{equation*}
with 
\begin{equation*}
    K_{mn} = - \sum_\mu \left[\frac{\delta_{m,n+\mu} + \delta_{m,n-\mu} - 2 \delta_{m,n}}{a^2}\right] + m_\phi^2 \delta_{mn} 
\end{equation*}
One can also express everything using dimensionless couplings
\begin{equation}
    \begin{aligned}
        \hat m_\phi^2 &= a^2 \, m_\phi^2 \\
        \hat \lambda &= a^{2} \, \lambda, \\
        \hat K_{mn} &= a^2 K_{mn}
    \end{aligned}
    \label{eq:couplings_redefitinion}
\end{equation}
and the action is then described only in terms of dimensionless quantities
\begin{equation*}
    S_\phi=-\sum_{n, \mu} \hat\phi_n \hat\phi_{n+\mu}+\sum_n\left[\frac{1}{2}\left(4+\hat m^2\right) \hat\phi_n^2 +\frac{\hat\lambda}{4 !} \, \hat\phi_n^4\right]
\end{equation*}
\textcolor{red}{Otherwise it is customary to introduce dimensionless couplings $\kappa, \beta$ defined via
\begin{equation}
    \begin{aligned}
       \phi & \rightarrow(2 k)^{\frac{1}{2}} \phi, \\
        (a m)^2 & \rightarrow \frac{1-2 \beta}{k}-4, \\
        a^{-2} \lambda & \rightarrow \frac{6 \beta}{k^2}
    \end{aligned}
    \label{eq:dimless_couplings_defitinion}
\end{equation}
and the equivalent action reads
\begin{equation*}
    S_{\phi}=-2 k \sum_{n, \mu} \phi_n^i \phi_{n+\mu}^i+(1-2 \beta) \sum_n \phi_n^i \phi_n^i+\beta \sum_n\left(\phi_n^i \phi_n^i\right)^2
\end{equation*}
In the following, we might use any of the two dimensionless formulations interchangeably, since they are completely equivalent given the definitions \eqref{eq:couplings_redefitinion}, \eqref{eq:dimless_couplings_defitinion}.} \\
For what concerns the fermionic action, a naive discretisation is not sufficient, due to the well known doubling problem. In this work Wilson fermions are employed as a way to fix such issue. Details of this formulation are explained in section \ref{AppendixB}. Here, only the final discretised action is reported, which reads
\begin{equation*}
    S_\psi[\phi, \psi, \bar\psi] = \sum_{m,n} \bar\psi_m \, D_{m,n} \psi_n + g \, \sum_n \bar\psi_n \phi_n \psi_n
\end{equation*}
with $\psi_n$ beeing a four-component spinor (2 flavour components and 2 Dirac components), and $D_{m,n}$ beeing the Wilson-Dirac operator \note{is $g\phi$ included in the definition of $D$?)} defined as 
\begin{equation}
    \begin{aligned}
    D_{m, n} = &- \left(\frac{\Gamma_{+0}}{2} \, \delta_{m, m+0} +\frac{\Gamma_{-0}}{2} \, \delta_{m, m-0} + \frac{\Gamma_{+1}}{2} \, \delta_{m, m+1} + \frac{\Gamma_{-1}}{2} \, \delta_{m, m-1}\right) \, \delta _{f, f'} \\
     &+ \left(2 + m + g\phi\right) \ \delta_{s,s'} \delta_{m,n} \\
    \end{aligned}
    \label{eq:wilson-dirac_operator}
\end{equation}
The Wilson projectors $\Gamma_{\pm \mu}$ are defined as
\begin{equation*}
    \Gamma_{\pm \mu} = 1 \mp \gamma_\mu 
\end{equation*}
One can then proceed by defining dimensionless fields and couplings
\begin{equation*}
    \begin{aligned}
        \hat\psi &= \rightarrow a^{\frac{1}{2}} \psi, \\
        \hat m_q &= a m_q, \\
        \hat g &= a g
    \end{aligned}
    \label{eq:fermionic_theory_dimless_redefinitions}
\end{equation*}
to describe the action only in terms of dimensionless quantities. \\
It is customary to perform the fermionic part of the path integral explicitly
\begin{equation*}
    \int \mathcal{D}\hat{\bar\psi} \mathcal{D}\hat\psi \ \exp\left( - \sum_{m,n} \hat{\bar\psi_m} \, \hat D_{mn} \, \hat\psi_n \right) = \text{det} \, \hat D[\hat\phi] = e^{\tr \log (\hat D[\hat\phi])}
\end{equation*}
where the trace is performed over space-time, flavour and spinor components. \\ 
The full path integral  reads 
\begin{equation*}
    Z = \int \mathcal{D}\hat\phi \ e^{-S_\text{eff}[\hat\phi]}
\end{equation*}
with the effective action
\begin{equation*}
    S_\text{eff}[\hat\phi] = S_{\phi}[\hat\phi] - \underset{x,s,f}{\tr} \log D[\hat\phi]
\end{equation*}
In the remainder of this work, the dimensionless couplings and fields will be adopted, unless otherwise specified. Additionally, both the original action $S$ and the effective action $S_\text{eff}$ will be denoted by $S$ for simplicity. It will be clear from the context to which of the two we will be referring.
\section{Continuum limit}
\label{sec:continuum_limit}
We briefly explain how the continuum limit is, in general, taken in LFT and the problem with effective theories, so that we can motivate the use of colored noise for the next section. \\~\\
We are constrained to the manifold of fixed $m^2_{\phi, 0}, m_{q,0}, g_0, \lambda_0$. We use $m_{q,r}$ to fix the scales for all the other dimensionful quantities. The cutoff is hence $1/{m_{q,r} a}$.
For perturbatively renormalisable theories, we know that we can fine tune the bare parameters (as a function of the cutoff?) to keep renormalised quantities finite as we take the continuum limit. Introduce renormalisation as a mapping as in page 40 of Montvay Munster.
We are interest in the set of theories in theories space that have constant renormalised couplings (trajectories in Kadanoff-Wilson RG) but different dimless masses $m_{q,r} \, a$. Since the latter gives the cutoff, we are moving the cutoff while keeping the other renormalised couplings fixed. How does this happen? We have to compensate the change in the cutoff by a change in the bare couplings as detailed here. IN perturbation theory we can compute renormalised couplings as functions of the bare ones. We can the (HOPEFULLY) invert these relations and determine bare parameters as functions of the renormalised ones. Near the continuum limit we can expect renormalised vertex functions to depend only weakly on the spacing so that we can impose
\begin{equation*}
    a \frac{d}{da} \Gamma_R^n(p^i, g^i_R, m_{q,r} \, a) = 0
\end{equation*}
And since 
\begin{equation*}
    \Gamma_R = Z \Gamma_0
\end{equation*}
one gets
\begin{equation*}
    \left(a\partial_a - \beta_\text{lat} \partial g_0 + n \gamma_\text{lat}\right) \ \Gamma_0^n = 0
\end{equation*}
Integrating these equations leaves us with functions
\begin{equation*}
    g_0^i(a)
\end{equation*}
that tell us how to change bare couplings as a back reaction to a change in the spacing in order to keep physics constant. \\
In this perspective we want renormalised fixed and we change bare ones. In the callan symanzik we keep bare fixed and we see how the renormalised change. How are the two beta functions related? See 1.271 in Montvay and Munster. In this way one gets the relation between the two beta functions. The normal one ca be computed in the continuum with standard renormalisation techniques, the lattice one deduced. They match till second order.