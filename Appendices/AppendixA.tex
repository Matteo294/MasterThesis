% !TeX root = ../main.tex
% Appendix A
 
\chapter{Conventions} % Main appendix title
\label{chap:AppendixA} % For referencing this appendix elsewhere, use \ref{AppendixA}
\subsection*{Spinor representation}
We choose the following representation of the two-dimensional $\gamma$ matrices
\begin{equation*}
        \gamma_0 = \sigma_3 = 
        \begin{pmatrix}
            1 & 0 \\
            0 & -1    
        \end{pmatrix} \, ,
        \qquad 
        \gamma_1 = \sigma_1 = 
        \begin{pmatrix}
            0 & 1 \\
            1 & 0    
        \end{pmatrix} \, .
\end{equation*}
They satisfy the Euclidean Clifford algebra
\begin{equation*}
    \{\gamma_\mu, \gamma_\nu\} = 2\delta_{\mu, \nu},
\end{equation*}
and they have the following properties
\begin{equation*}
    \gamma_\mu^{\dagger}=\gamma_\mu, \quad  \quad \gamma_\mu^2=\mathbb{1}.
\end{equation*}
We then introduce $\gamma_5$ as
\begin{equation*}
    \gamma_5 = i \gamma_0 \gamma_1 =
    \begin{pmatrix}
        0 & i \\
        -i & 0
    \end{pmatrix},
\end{equation*}
with the following properties
\begin{equation*} 
    \gamma_5^{\dagger}=\gamma_5, \qquad \gamma_5^2=\mathbb{1}, \qquad \{\gamma_5, \gamma_\mu\} = 0 \, .
\end{equation*}



\subsection*{Fourier transform}
The Fourier transformation of a function $f: \mathbb{R}^d \rightarrow \mathbb{R}$ is given by
\begin{equation*}
\bar{f}(p)=\mathcal{F}[f](p)=\int_x e^{-ixp} f(x) \, ,
\end{equation*}
and its inverse is
\begin{equation*}
f(x)=\mathcal{F}^{-1}[\bar{f}](x)=\frac{1}{(2 \pi)^d} \int_p e^{ixp} \bar{f}(p).
\end{equation*}
