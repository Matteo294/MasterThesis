% !TeX root = ../main.tex
\chapter{Introduction and outline}
\label{chap:introduction}

\section{Quantum chromodynamics and its and phase diagram}
Big picture: here we talk about QCD and the problem of the phase diagram
\section{Effective theories}
Here we first define effective theories and discuss their usefulness, then introduce RG as a technique to resolve physics at different scales. \\
\vspace{20pt}
Motivation for choices, connection to stochastic regularisation, complex langevin, easy noise \cite{boo}
Motivation for choices, \\ ~
The starting set up is the euclidean formulation of quantum field theory, where one typically defines a path integral $Z$, which, for a general scalar field $\phi(x)$ and a fermion field $\psi(x)$, assumes the form
\begin{equation}
    Z = \int \mathcal{D}\phi\mathcal{D}\psi\mathcal{D}\bar\psi \ e^{-S[\phi, \psi, \bar\psi]} \qquad \mathcal{D}\xi = \prod_x d\xi_x, \quad \xi \in \{\phi, \psi, \bar\psi\}
    \label{eq:path_integral_generic}
\end{equation}
One then aims at computing correlation functions via
\begin{equation*}
        \left\langle \xi_{x_1} \dots \xi_{x_n}  \right\rangle = \frac{1}{Z} \int \mathcal{D}\phi\mathcal{D}\psi\mathcal{D}\psi \ \xi_{x_1} \dots \xi_{x_n} \ e^{-S[\phi, \psi, \bar\psi]} \qquad \xi_{x_i} \in \{\phi_{x_i}, \psi_{x_i}, \bar\psi_{x_i}\} \\
\end{equation*}
Computing physical quantities from such a direct approach is not only hard to do, but results often impossible due to the appearence of divergences in the calculations. To fix this, one often relies on expansion techniques such as perturbation theory (\textcolor{red}{CITATION}), in which one tries to regularise the theory order by order in an expansion on the interaction coupling, yielding finite quantities that depend on the truncation order.  While this method is capable of producing incredibly precise results (\textcolor{red}{g-2, fine structure, ...}), it fails completely in treating non-perturbative phenomena, namely effects that cannot be captured by any order in the expansion or that are typical of strongly interacting systems. Example of such systems range from \textcolor{red}{QCD, cold atoms, plasma, stuff}. 
Moreover, such formulation is also not much suitable for numerical computations, since both the path integral and action measure are infinite dimensional objects. \\
Lattice field theory \cite{Montvay1994QuantumLattice,rothe_LGT,gattringer_LQCD,creutz_2023} is meant at first as a powerful non-perturbative regularisation tool to prevent divergences to occur and render the computation of the correlation functions finite. Moreover, it also provides a framework to study quantum field theory numerically on a computer. In order to accomplish this, one typically defines the theory on a space-time lattice and makes use of statistical methods such as Monte Carlo algorithms to compute observables. One may wonder how can one reconstruct the results in the continuum theory, keeping the results finite, and matching the results on the discretised theory to physically measured ones. This task, far from beeing simple, will be the focus of the next sections, in which we will first introduce relevant theoretical tools, such as the renormalisation group, and then discuss the existence of a continuum limit of a lattice theory and, if it exists, how it can be extracted. This will motivate the introduction of coloured noise in the context of continuum limits of effective theories, a technique which will be shown to be powerful also for other various reasons, which will be the main focus of the analysis carried in the remaining chapters.
