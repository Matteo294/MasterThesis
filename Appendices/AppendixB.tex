% !TeX root = ../main.tex
\chapter{Wilson fermions}
\label{chap:AppendixB}
Fermion Dirac 
\begin{equation*}
    \widehat D = \hat m_q - i \sum_\mu \gamma_\mu \sin \left(p_\mu \, a\right)
\end{equation*}
which can be easily inverted
\begin{equation*}
    \widehat{D}^{-1}(p) = \frac{\hat m_q - i \sum_\mu \gamma_\mu \sin \left(p_\mu \, a\right)}{\hat m_q^2 + \sum_\mu \sin^2 \left(p_\mu \, a \right)}
\end{equation*}
Doublers at the edges of the Brillouin zone $\hat p_\mu \in \{\pm \pi/a, \pm \pi/a\}$ \\
Wilson's idea:
\begin{equation*}
    S_{W} = S - \frac{r}{2} \, \sum_{m,n} \hat{\bar\psi}(m) \ \hat\Box \ \hat{\psi}(n)
\end{equation*}
where 
\begin{equation*}
    \left(\hat\Box\right)_{mn} = \sum_{\hat\mu} \ \frac{\delta_{m, n+\hat\mu} + \delta_{m, n-\hat\mu} - 2 \, \delta_{mn}}{2}
\end{equation*}

\begin{equation*}
    \begin{aligned}
        \bar{D}^{-1}(p) = \frac{m_q + \frac{2r}{a} \, \sum_\mu \sin ^2\left(\frac{p_\mu a}{2}\right) - i \sum_\mu \gamma_\mu \sin \left(p_\mu a\right)}{\left[m_q + \frac{2r}{a} \, \sum_\mu \sin^2\left(\frac{p_\mu a}{2}\right)\right]^2 + \sum_\mu \sin^2 \left(p_\mu a\right)} = \frac{M(p) - i \sum_\mu \gamma_\mu \sin \left(p_\mu a\right)}{M^2(p) + \sum_\mu \sin^2 \left(p_\mu a\right)}
    \end{aligned}
\end{equation*}
For every $\hat p_\mu \not\in \{\pm \pi/a, \pm \pi/a\}$, one has that $a \to 0$ implies $M(p) \to m_q$. Instead, for every momentum vector that lies at the edge of the Brillouin zone $M(p) \to \infty$ as $a \to 0$. This removes the fermion doublers in the continuum limit. \\
Thus, one loses chiral symmetry since the Wilson term is clearly not invariant under neither the continuous transformation
\begin{equation*}
    \psi \to e^{i \theta \gamma_5} \, \psi \qquad \bar\psi \to - \bar\psi \, e^{i \theta \gamma_5},
\end{equation*}
nor the discrete
\begin{equation*}
    \psi \to \gamma_5 \, \psi \qquad \bar\psi \to - \bar\psi \, \gamma_5.
\end{equation*}