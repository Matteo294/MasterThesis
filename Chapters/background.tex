% !TeX root = ../main.tex
\chapter{Theoretical background}
\label{chap:theoretical_background}

\section{Lattice formulation of Euclidean Quantum Field Theory}
\label{sec:lattice_formulation}
In the euclidean formulation of quantum field theory, one typically defines the path integral as 
\begin{equation}
    Z = \int \mathcal{D}\phi\mathcal{D}\psi\mathcal{D}\psi \ e^{-S[\phi, \psi, \bar\psi]}
    \label{eq:path_integral_generic}
\end{equation}
and aims to compute correlation functions as
\begin{equation*}
        \left\langle \xi_{x_1} \dots \xi_{x_n}  \right\rangle = \frac{1}{Z} \int \mathcal{D}\phi\mathcal{D}\psi\mathcal{D}\psi \ \xi_{x_1} \dots \xi_{x_n} \ e^{-S[\phi, \psi, \bar\psi]} \qquad \xi_{x_i} \in \{\phi_{x_i}, \psi_{x_i}, \bar\psi_{x_i}\} \\
\end{equation*}

\section{Yukawa theory}
\label{sec:Yukawa_theory}
Let us consider the Yukawa theory defined by the action
\begin{equation}
\begin{aligned}
    S[\phi,\psi,\bar\psi] &= S_\phi[\phi] + S_\psi[\psi, \bar\psi] + S_\text{int}[\phi, \psi, \bar\psi]\\
     S_\phi[\phi] &= \int_x \phi_x \left(-\frac{\partial^2_x}{2} + \frac{m_\phi^2}{2}\right) \phi_x + \frac{\lambda}{4!} \, \phi_x^4 \\
     S_\psi[\phi, \psi, \bar\psi] &= \int_x \bar \psi_x \left(\slashed{\partial}_x + m_q \right) \psi_x \\
     S_\text{int}[\phi, \psi, \bar\psi] &= g \, \int_x \bar \psi_x \, \phi_x \, \psi_x
    \label{eq:full_action_continuum}
\end{aligned}
\end{equation}
One can see that the action is made of a scalar part $S_\phi[\phi]$, a fermionic part $S_\psi[\psi, \bar\psi]$ and a Yukawa interaction term $S_\text{int}[\phi, \psi, \bar\psi]$. \\
It is also convenient for later purposes to define the operators $K, D$ represented in position space as 
\begin{equation}
    \begin{aligned}
        K(x, y) &=  \left(-\frac{\partial^2_x}{2} + \frac{m_\phi^2}{2}\right) \ \delta(x,y) \\
        D(x, y) &= \left(\slashed{\partial}_x + m_q + g\phi \right) \ \delta(x,y)
    \end{aligned}
    \label{eq:definition_kinetic_terms_continuum}
\end{equation}
and in momentum space as
\begin{equation}
    \begin{aligned}
        \widetilde{K}(p, q) &=  \int_{x,y} e^{-ipx} \left(-\frac{\partial^2_x}{2} + \frac{m_\phi^2}{2}\right) \ \delta(x,y) \ e^{iqy} = \left(\frac{p^2}{2} + \frac{m_\phi^2}{2}\right) \ \delta(p,q) \\
        \widetilde{D}(p, q) &= \int_{x,y} e^{-ipx} \left(\slashed{\partial}_x + m_q + g\phi \right) \ \delta(x,y) \ e^{iqy} = \left(\slashed{p}_x + m_q + g\phi \right) \ \delta(p,q)
    \end{aligned}
    \label{eq:definition_kinetic_terms_continuum}
\end{equation}
This allows one to rewrite the action as
\begin{equation*}
    S[\phi,\psi,\bar\psi] = \int_x \phi_x K \phi_x + \frac{\lambda}{4!}\phi_x^4 + \bar\psi_x D \psi_x
\end{equation*}
For vanishing quark mass the action is fully invariant under the chiral transformation
\begin{equation}
    \begin{aligned}
        \phi &\to -\phi \\
        \psi &\to e^{i\alpha} \, \psi \\
        \bar\psi &\to \bar\psi \, e^{-i\alpha} \\
    \end{aligned}
    \label{eq:yukawa_continuum_rewritten}
\end{equation}
The main feature of the model is chiral symmetry breaking \cite{Nambu1961DynamicalI, Nambu1961DynamicalII}, which can happen explicitly at the level of the classical action for a non-zero quark mass, or spontaneously when the scalar field gains a non-zero expectation value. One can in fact notice already by looking at \eqref{eq:definition_kinetic_terms_continuum}, that $\left\langle\phi\right\rangle \neq 0$ has the same effect on the action as a finite bare quark mass. This observation will be made more quantitative in section (SECCCC) where it will be shown that  
\begin{equation*}
    \left\langle \phi \right\rangle \sim \left\langle \bar \psi \psi \right\rangle \sim D^{-1}
\end{equation*}
The fermionic part of the path integral \eqref{eq:path_integral_generic} can be performed explicitly
\begin{equation*}
    \int \mathcal{D} \bar\psi \mathcal{D} \psi \ \exp\left( - \int_x \bar\psi_x \,  D \, \psi_x \right) = \text{det} \, D[\phi] = e^{\tr \log (D[\phi])}
\end{equation*}
where the trace is performed over space-time, flavour and spinor components. \\ 
The full path integral can now be expressed in terms of the resulting effective action for the scalar fields
\begin{equation*}
    Z = \int \mathcal{D}\phi \ e^{-S_\text{eff}[phi]}
\end{equation*}
with
\begin{equation}
    S_\text{eff}[\phi] = S_{\phi}[\phi] - \underset{x,s,f}{\tr} \log D[\phi]
    \label{eq:effective_action_no_fermions}
\end{equation}
One can derive the classical equations of motion by imposing $\frac{\delta S}{\delta \phi} = 0$, here expressed in momentum space
\begin{equation*}
     (k^2 + m_\phi^2) \, \phi(x) + \frac{\lambda}{6} \, \phi^3(x) = g \ \underset{s,f}{\tr} \left[D^{-1}(\phi(x)\right] = - g \ \bar\psi(x) \psi(x)
\end{equation*}
where the trace is performed over spin and flavour components. For $\lambda = 0$, they highlight a simple proportionality relation between magnetisation and chiral condensate, which for zero momentum reads
\begin{equation}
    \phi(x) = - \frac{g}{m_\phi^2} \ \bar \psi(x) \psi(x)
    \label{eq:classical_EOM}
\end{equation}
The classical relation \eqref{eq:classical_EOM} is proven to hold also at mean field on the quantum level \cite{Buballa2005NJL-modelMatter} and will be studied in the discretised theory in section \ref{sec:classical_to_quantum}. \\~\\


\begin{figure}[h]
\centering
\begin{minipage}{0.45\textwidth}
    \begin{tikzpicture}
        \begin{axis} [axis lines=center, xtick=\empty, ytick=\empty, xlabel=$\phi$, ylabel=$V(\phi)$,
        every axis x label/.style={
            at={(ticklabel* cs:1.0)},
            anchor=west,
        },
        every axis y label/.style={
            at={(ticklabel* cs:1.0)},
            anchor=south,
        },]
            \addplot [domain=-3:3, smooth, thick] { 6*x^2 + x^4 - 6*x };
        \end{axis}
     
    \end{tikzpicture}
    
\end{minipage}
\hfill
\begin{minipage}{0.45\textwidth}
    \begin{tikzpicture}
        \begin{axis} [axis lines=center, xtick=\empty, ytick=\empty, xlabel=$\phi$, ylabel=$V(\phi)$,
        every axis x label/.style={
            at={(ticklabel* cs:1.0)},
            anchor=west,
        },
        every axis y label/.style={
            at={(ticklabel* cs:1.0)},
            anchor=south,
        },]
            \addplot [domain=-3:3, smooth, thick] { -6*x^2 + x^4 - 2*x };
        \end{axis}
     
    \end{tikzpicture}
\end{minipage}
\label{fig:breaking_O1_symmetry}
\caption{The introduction of the boson-fermion interaction, with a finite fermionic mass, causes the breaking of the O(1) symmetry. It shifts the equilibrium position in the symmetric phase (left) causing $\left\langle \phi \right\rangle = 0$, and tilts the potential in the broken phase (right), making the two minima not equivalent.}
\end{figure}

\newpage


\section{Block spin RG}
\label{sec:blockspin}
Cite Kadanoff article \cite{PhysicsPhysiqueFizika.2.263}. Average spins and rescale stuff to keep correlation length fixed.

\section{Wilson RG}
\label{sec:wilson_rg}
Extends block spin RG. \\
One splits the fields as $\Phi = \phi + \varphi$ where $\phi$ are fields with momenta $p \leq b\Lambda$ and $\varphi$ are fields with momenta $b\Lambda < p < \Lambda$, then one writes the path integral in terms of the Wilsonian effective action
\begin{equation*}
    Z = \int D\Phi_\Lambda \, e^{-S_\Lambda[\Phi]} = \int D\phi_{b\Lambda} \, e^{-S_{b\Lambda}[\phi]} \int D\varphi_{b\Lambda, \Lambda}  \, e^{-S_{b\Lambda, \Lambda}[\phi, \varphi]} = \int D\phi_{b\Lambda} \, e^{-S_{b\Lambda}^\text{eff}[\phi]}
\end{equation*}
where 
\begin{equation*}
    S_{b\Lambda}^\text{eff}[\phi] = S_{b\Lambda}[\phi] - \log\left( \int D\varphi_{b\Lambda, \Lambda}  \, e^{-S_{b\Lambda, \Lambda}[\phi, \varphi]}\right) =  S_{b\Lambda}[\phi] + \Delta S_{b\Lambda, \Lambda}[\phi]
\end{equation*}
Note that all the steps above are exact identities. In particular, performing the integral over ultraviolet modes, is the continuum version of the block spinning procedure outlined in the previous section. \\
Note also that $S_{b\Lambda}[\phi]$ is the same as the initial action, but it is non-zero only for fields with $p^2 \leq b\Lambda^2$. \\
At this point one can expand $\Delta S_{b\Lambda, \Lambda}$ in powers of the field \textcolor{red}{(before, another step, see jan pawlowski's notes)}. Powers that are present also in $S_{b\Lambda}$ can be absorbed into the latter by redefining the coupling
\begin{equation}
\begin{aligned}
& \phi^{\prime}\left(x^{\prime}\right)=\left[b^{2-d}(1+\Delta z)\right]^{\frac{1}{2}} \phi(x), \quad m^{\prime 2}=\left(m^2+\Delta m^2\right) \frac{1}{1+\Delta z} \frac{1}{b^2}, \quad \lambda^{\prime}=(\lambda+\Delta \lambda) \frac{1}{(1+\Delta z)^2} b^{d-4}, \\
& \alpha^{\prime}=(\alpha+\Delta \alpha) \frac{1}{(1+\Delta z)^2} b^d, \quad \lambda_6^{\prime}=\left(\lambda_6+\Delta \lambda_6\right) \frac{1}{(1+\Delta z)^3} b^{2 d-6}, \quad \ldots \\
&
\end{aligned}
\end{equation}
higher powers are suppressed (non-renormalisable terms) are suppressed. By neglecting these higher order powers, one can bring the action in the same form as the initial one via redefinition of the parameters. The whole procedure is non-perturbative. The result can be compared to the initial action after redefinition of all the dimensionful quantities i.e. via $p'\equiv p/b$. 

\section{Continuum limit}
\label{sec:continuum_limit}
Introduce renormalisation as a mapping as in page 40 of Montvay Munster. \\
Continuum limits in lattice theories are intimately connected to the existence of critical points in the theories. In fact, to take a continuum limit, one want the dimensionless correlation length $\hat \xi$ to diverge: in this way one can represent an infinite number of points inside a finite volume (\textcolor{red}{explain better here}). For this to happen, the system must go under a second order phase transition, whose critical point is identified by a set of values for the bare parameters $g_0^{i*}$. \textcolor{red}{differentiate well between dimful and dimless}. \\
Consider $O$ to be an observable which has to be matched to a physical measurable quantity, and compare it to the dimensionless quantity $\hat O$ given by a lattice simulation. In general the physical observable is assumed to be a function of the spacing and the bare couplings of the theory 
\begin{equation*}
    O = O(a, g^i_0)
\end{equation*}
while its lattice counterpart can only depend on the dimensionless coupling $hat g_0^i$, i.e.
\begin{equation*}
    \hat O = \hat O(g_0^i)
\end{equation*}
Let $d_O$ be the physical dimension of the observable $O$ in units of energy. Then one can relate the two quantities as 
\begin{equation}
    O(a, g^i_0) = \left(\frac{1}{a}\right)^{d_O} \hat O(\hat g_0^i)
    \label{eq:physical_observable_lattice_observable}
\end{equation}
We now want to address the following question: given a (small enough) a, is it possible to find a value $\hat g_0^i(a)$ such that the value of $O$ given by \eqref{eq:physical_observable_lattice_observable} does not depend on $a$? \\
We then impose such condition via 
\begin{equation*}
    \frac{d}{da} \, O(a, g^i_O) = \left(a\frac{\partial}{\partial a} - \beta(g_0^i) \frac{\partial}{\partial g_0^i}\right) \, O(a, g^i_O) = 0
\end{equation*}
with
\begin{equation*}
    \beta(g_0^i) = - a \frac{\partial g_0^i}{\partial a}
\end{equation*}
Integrating such $\beta$ functions tells one how to change bare couplings as a backreaction to a change in the spacing, in order to keep observables constant. We then say that the theory admits a continuum limit if there exists some set of values $(g_0^i)^*$ such that when $g_0^i \to (g_0^i)^*$ one has $\hat \xi \to +\infty$ and $O \to O_{phys}$. \textcolor{red}{Connection with fixed points and beta function. Comment also on beta functions for dimless couplings}. \\
Of course one does not know a priori the full lattice beta functions, but they can be computed via approximate or continuum methods. For example, in continuum perturbation theory, one can compute $g_r^i(\Lambda) = g_r^i(\Lambda, g_0^j)$, where $\Lambda$ is a sharp momentum cutoff and then try to invert them to find $g_0^i = g_0^i(\Lambda, g_r^j)$. The connection is then given by making the identification $a \sim \Lambda^{-1}$. \\
Generally speaking, we are interested in the set of theories in theories space that have constant renormalised couplings but different dimless couplings $g_r^i \, a$ (trajectories in Kadanoff-Wilson RG).