% !TeX root = ../main.tex
\chapter{Theoretical background}
\label{chap:background}

Da qualche parte cita \cite{carosso2020novel}. \\~\\
In this chapter we want to provide with an overview on the general theoretical framework that supports this work, and introduce the main concepts for the successive parts. Each section in this chapter is, by no means, meant as an exhaustive treatment. The description will be quite conceptual, rather than technical, and aims at recalling the main ideas and fix conventions. We ask the reader to consult appropriate references,  which will be given in the corresponding sections, for a more detailed treatment of the topics. \\~\\
The starting set up is the euclidean formulation of quantum field theory, where one typically defines a path integral $Z$
\begin{equation}
    Z = \int \mathcal{D}\phi\mathcal{D}\psi\mathcal{D}\psi \ e^{-S[\phi, \psi, \bar\psi]}
    \label{eq:path_integral_generic}
\end{equation}
and aims to compute correlation functions
\begin{equation*}
        \left\langle \xi_{x_1} \dots \xi_{x_n}  \right\rangle = \frac{1}{Z} \int \mathcal{D}\phi\mathcal{D}\psi\mathcal{D}\psi \ \xi_{x_1} \dots \xi_{x_n} \ e^{-S[\phi, \psi, \bar\psi]} \qquad \xi_{x_i} \in \{\phi_{x_i}, \psi_{x_i}, \bar\psi_{x_i}\} \\
\end{equation*}
Computing physical quantities from such a direct approach is not only hard to do, but results often impossible due to the appearence of divergences in the calculations. To fix this, one often relies on expansion techniques such as perturbation theory (\textcolor{red}{CITATION}), in which one tries to regularise the theory order by order in an expansion on the interaction coupling, yielding finite quantities that depend on the truncation order.  While this method produced incredibly precise results (\textcolor{red}{g-2, fine structure, ...}), it fails completely in treating non-perturbative phenomena, namely effects that cannot be captured by any order in the expansion or that rely on strongly interacting matter. Example of such systems range from \textcolor{red}{QCD, cold atoms, plasma, stuff}. 
Moreover, such formulation is also not much suitable for numerical computations, since both the path integral and action measure are infinite dimensional objects. \\
Lattice field theory \cite{Montvay1994QuantumLattice,rothe_LGT,gattringer_LQCD,creutz_2023} is meant at first as a powerful non-perturbative regularisation tool to prevent divergences to occur and render the computation of the correlation functions finite. Moreover, it also provides a framework to study quantum field theory numerically on a computer. In order to accomplish this, one typically defines the theory on a space-time lattice and makes use of statistical methods such as Monte Carlo algorithms to compute observables. One may wonder how can one reconstruct the results in the continuum theory, keeping the results finite, and matching the results on the discretised theory to physically measured ones. This task, far from beeing simple, will be the focus of the next sections, in which we will first introduce relevant theoretical tools, such as the renormalisation group, and then discuss the existence of a continuum limit of a lattice theory and, if it exists, how it can be extracted. This will motivate the introduction of coloured noise in the context of continuum limits of effective theories, a technique which will be shown to be powerful also for other various reasons, which will be the main focus of the analysis carried in the remaining chapters.

\section{Lattice Field Theory}
\label{sec:lattice_formulation}
As mentioned before, lattice field theory consists in formulating QFT on a space-time lattice
\begin{equation*}
	S = \int d^dx \, \mathcal{L}(x) \ t
\end{equation*}

\section{Block-spin renormalisation}
\label{sec:blockspin}
Landau's mean field approach to study phase transition \cite{Landau:1937obd}, soon proved to be inaccurate to predict some experimentally well proven properties of certain systems around the critical point \textcolor{red}{add some examples}. This is because, beeing a mean field theory, it did not take into account spatial fluctuations of the underlying quantities.
The idea of block-spin transformation developed mainly by Kadanoff \cite{PhysicsPhysiqueFizika_2_263} made a big step towards a deeper understanding of the scaling behaviour, and posed the basis for the later work of Wilson \cite{WilsonRG1,WilsonRG2,WilsonFisher}, which still constitutes the basis for modern approaches to renormalisation in field theory and statistical physics.\\
To illustrate the idea, let us consider a set of spins whose magnetisation is described by a function $\varphi(x)$. If the spins are on a discrete lattice $\mathcal{L}$, then the function assumes finite values only at such sites $\varphi(x_i) = \phi_i \neq 0 \Leftrightarrow x_i \in \mathcal{L}$. Suppose then that their interaction is described by a certain action $S[\varphi]$ such as 
\begin{equation*}
    S[\phi] =  \sum_{n} - J\sum_{\hat\mu} \left(\phi_n \, \phi_{n+\hat\mu}\right) + b \, \phi^2_n + \lambda \, \phi^4_n
    \label{eq:example_action_blockspin}
\end{equation*}
which yield the partition function
\begin{equation*}
    Z=\sum_{\varphi} \mathrm{e}^{-S[\varphi]}
\end{equation*}
We now want to introduce a coarse-grained (or blocked) field $\bar\varphi$, which can be, for example, defined as
\begin{equation*} 
    \bar\varphi(x_i) = \Delta \, \sum_{\hat\mu} \varphi(x_i + \hat \mu)
\end{equation*}
and we now want to find a new action $S_b$ such that 
\begin{equation}
    Z=\sum_{\varphi} \mathrm{e}^{-S(\varphi)}= \sum_{\bar\varphi} \mathrm{e}^{-S_b\left(\bar\varphi\right)}
\end{equation}
We can always, in principle, cook up such an action, but it is complicated. In fact, after a block-spin transformation such as the above, it is in general not possible to recover the original partition function but simply redefining the parameters that enter the action. By looking at \eqref{eq:example_action_blockspin} it is easy to get convinced that, a description in terms of the field $\bar\phi$ will produce higher order interactions such as nearest-to-nearest neighbour, and the new action should, in principle, contatins newer terms (all the terms compatible with original symmetries). \textcolor{red}{mention that it should contain all the terms compatible with the symmetries and cite some papaers.}\\
Fortunately, higher order terms are smaller and can be (initially) neglected \textcolor{red}{add citation}. \\
Note that the coarse-graining procedure comes with a loss of resolution since the spacing is changed $a \to a/s$. This means that there is a loss of information and the system is not anymore capable of describing physics at scale below the new spacing. In particular, this means also that the dimensionless correlation length is reduced $\hat \xi \to \hat \xi' = \xi / a' = \hat\xi /2$. Hence, for a comparison with the original action, one has to rescale again all the dimensionfull quantities as pictured in figure \ref{fig:blockspin}. The procedure of rescaling can be thought as a "zooming out", for an inifinite volume system. \\
The philosophy of both Kadanoff
and Wilson was that the blocking transformation therefore reduces the complexity of many-body
systems by systematically reducing the number of degrees of freedom being taken into account,
without changing the physics \cite{WILSON197475} (because the partition function is invariant). \textcolor{red}{Because critical phenomena are characterized by large
correlation lengths, block-spin RG proves to be a useful tool\\
s. From $a = \xi / \hat\xi$, we also see that the critical fixed points correspond to zero lattice
spacing systems (if $\xi \neq 0$), i.e. the continuum limit.}

\section{Wilsonian renormalisation}
\label{sec:wilson_rg}
The wilsonian picture of renormalisation \cite{WilsonRG1,WilsonRG2} is formulated in momentum space and in general more suitable for theories in the continuum.\\
The idea is that a physical theory comes with an intrinsic cutoff which defines up to which scale the theory is valid. This theory, which is considered fundamental, might not be the one
we observe directly in experiments at energy scales $\Lambda'$, which we call effective theory at scale $\Lambda'$. To see how can this happen, let us consider a fundamental theory defined by an action $S_\Lambda[\Phi]$, and let us split the field as $\Phi = \phi + \varphi$ where $\phi$ are fields with momenta $p \leq \Lambda'$ and $\varphi$ are fields with momenta $\Lambda' < p \leq \Lambda$. This allows one to rewrite the path integral as
\begin{equation*}
    Z = \int D\Phi_\Lambda \, e^{-S_\Lambda[\Phi]} = \int D\phi_{\Lambda'} \, e^{-S_{\Lambda'}[\phi]} \int D\varphi_{\Lambda', \Lambda}  \, e^{-S_{\Lambda', \Lambda}[\phi, \varphi]} = \int D\phi_{\Lambda'} \, e^{-S_{\Lambda'}^\text{eff}[\phi]}
\end{equation*}
where, in the last step, we formally performed the integral over the high-momentum field $\varphi$ and defined a new effective action $S^\text{eff}_{\Lambda'}$ as
\begin{equation*}
    S_{\Lambda'}^\text{eff}[\phi] = S_{\Lambda'}[\phi] - \log\left( \int D\varphi_{\Lambda', \Lambda}  \, e^{-S_{\Lambda', \Lambda}[\phi, \varphi]}\right) =  S_{\Lambda'}[\phi] + \Delta S_{\Lambda', \Lambda}[\phi]
\end{equation*}
Note that all the steps above are exact identities. At this point, one typically tries to compute $\Delta S_{\Lambda', \Lambda}$ in some approximation and expand the result in terms of local operators $O_{m,n} = \frac{\partial^m}{\partial x^m}\phi^n(x)$ with respective coefficiends $g_{mn}$. This will generate many terms, some which were already present in the original action $S_\Lambda$ (and hence in $S_{\Lambda'}$), while some others will be new. The former can be readily absorbed in $S_{\Lambda'}[\phi]$ by a redefitinion of the couplings, while the latter are (at least initially) neglected, as they are UV-suppressed \cite{SOMEONE}. By neglecting these higher order powers, one can bring the action in the same form as the initial, upon redefinition of the parameters. Hence all the quantum fluctuations with momenta $p > \Lambda'$ are effectively encoded in the the new couplings. \\
Note that the new action is not anymore capable of describing degrees of freedom above the scale $\Lambda'$, and we lost resolution. At this point, one can clearly see the analogy with the block-spin transformation introduced in the previous section. Performing the integral over high momenta modes can be thought as performing averages (coarse graining) over neighbours. This causes a loss of resolution which can be recovered by rescaling dimensionful quantities, which can be pictured as a zoom out. Let us now define a dimensionless parameter 
\begin{equation*}
    s = \frac{\Lambda'}{\Lambda} \qquad 0 \leq s \leq 1
\end{equation*}\\
Combining the coarse-graining effects due to the high-modes integral and dimensional rescaling, the Wilson step causes a change in each coupling which can be schematically written as
\begin{equation*}
    g'_{mn}(s) = s^{[g_{mn}]} \frac{(1 + \delta g_{mn})}{Z_\phi^{nd_\phi/2}} \ g_{mn}
\end{equation*}
where $s^{[g_{mn}]}$ appears because of the dimensional rescaling mentioned above, $\delta g$ is the correction term that comes from the expansion of the high-momentum integral, and $Z_\phi$ is due to the fact that the high-momentum integral can generate corrections to the kinetic term, which we want to keep always the same by re-defining the field, hence the appearence of such factor. \\
Successive iterations of the above procedure gives rise to a flow of the couplings as functions of the scaling parameter $s$. Note that the assumption in which new terms in the action can be neglected is highly non-trivial, and in general higher order iterations must take into account these corrections. \\ 
The rescaling defined above then motivates the following classification for any operators entering the action
\begin{itemize}
    \item relevant 
    \item irrelevant
    \item marginal
\end{itemize}
Flow of the couplings, flow of the dimensionless couplings. Fixed points and linear behaviour around gaussian fixed point. Dimensionfull couplings flow is dominated by canonical dimension. FRG to show the connection with regularised noise.


\section{Continuum limit of lattice field theories}
\label{sec:continuum_limit}
Introduce renormalisation as a mapping as in page 40 of Montvay Munster. \\
Continuum limits in lattice theories are intimately connected to the existence of critical points in the theories. In fact, to take a continuum limit, one want the dimensionless correlation length $\hat \xi$ to diverge: in this way one can represent an infinite number of points inside a finite volume (\textcolor{red}{explain better here}). For this to happen, the system must go under a second order phase transition, whose critical point is identified by a set of values for the bare parameters $g_0^{i*}$. \textcolor{red}{differentiate well between dimful and dimless}. \\
Consider $O$ to be an observable which has to be matched to a physical measurable quantity, and compare it to the dimensionless quantity $\hat O$ given by a lattice simulation. In general the physical observable is assumed to be a function of the spacing and the bare couplings of the theory 
\begin{equation*}
    O = O(a, g^i_0)
\end{equation*}
while its lattice counterpart can only depend on the dimensionless coupling $hat g_0^i$, i.e.
\begin{equation*}
    \hat O = \hat O(g_0^i)
\end{equation*}
Let $d_O$ be the physical dimension of the observable $O$ in units of energy. Then one can relate the two quantities as 
\begin{equation}
    O(a, g^i_0) = \left(\frac{1}{a}\right)^{d_O} \hat O(\hat g_0^i)
    \label{eq:physical_observable_lattice_observable}
\end{equation}
We now want to address the following question: given a (small enough) a, is it possible to find a value $\hat g_0^i(a)$ such that the value of $O$ given by \eqref{eq:physical_observable_lattice_observable} does not depend on $a$? \\
We then impose such condition via 
\begin{equation*}
    \frac{d}{da} \, O(a, g^i_O) = \left(a\frac{\partial}{\partial a} - \beta(g_0^i) \frac{\partial}{\partial g_0^i}\right) \, O(a, g^i_O) = 0
\end{equation*}
with
\begin{equation*}
    \beta(g_0^i) = - a \frac{\partial g_0^i}{\partial a}
\end{equation*}
Integrating such $\beta$ functions tells one how to change bare couplings as a backreaction to a change in the spacing, in order to keep observables constant. We then say that the theory admits a continuum limit if there exists some set of values $(g_0^i)^*$ such that when $g_0^i \to (g_0^i)^*$ one has $\hat \xi \to +\infty$ and $O \to O_{phys}$. \textcolor{red}{Connection with fixed points and beta function. Comment also on beta functions for dimless couplings}. \\
Of course one does not know a priori the full lattice beta functions, but they can be computed via approximate or continuum methods. For example, in continuum perturbation theory, one can compute $g_r^i(\Lambda) = g_r^i(\Lambda, g_0^j)$, where $\Lambda$ is a sharp momentum cutoff and then try to invert them to find $g_0^i = g_0^i(\Lambda, g_r^j)$. The connection is then given by making the identification $a \sim \Lambda^{-1}$. \\
Generally speaking, we are interested in the set of theories in theories space that have constant renormalised couplings but different dimless couplings $g_r^i \, a$ (trajectories in Kadanoff-Wilson RG).

\section{Yukawa theory}
\label{sec:Yukawa_theory}
Let us consider the Yukawa theory defined by the action
\begin{equation}
\begin{aligned}
    S[\phi,\psi,\bar\psi] &= S_\phi[\phi] + S_\psi[\psi, \bar\psi] + S_\text{int}[\phi, \psi, \bar\psi]\\
     S_\phi[\phi] &= \int_x \phi_x \left(-\frac{\partial^2_x}{2} + \frac{m_\phi^2}{2}\right) \phi_x + \frac{\lambda}{4!} \, \phi_x^4 \\
     S_\psi[\psi, \bar\psi] &= \int_x \sum_{f=1}^{N_f} \bar \psi_x^{(f)} \left(\slashed{\partial}_x + m_q \right) \psi_x^{(f)} \\
     S_\text{int}[\phi, \psi, \bar\psi] &= \int_x \sum_{f=1}^{N_f} g \, \bar \psi_x^{(f)} \, \phi_x \, \psi_x^{(f)}
    \label{eq:full_action_continuum}
\end{aligned}
\end{equation}
One can see that the action is made of a scalar part $S_\phi[\phi]$, a fermionic part $S_\psi[\psi, \bar\psi]$ and a Yukawa interaction term $S_\text{int}[\phi, \psi, \bar\psi]$. \\
It is also convenient for later purposes to define the operators $K, D$ represented in position space as 
\begin{equation}
    \begin{aligned}
        K(x, y) &=  \left(-\partial^2_x + m_\phi^2\right) \ \delta(x,y) \\
        D(x, y) &= \left(\slashed{\partial}_x + m_q + g\phi \right) \ \delta(x,y)
    \end{aligned}
    \label{eq:definition_kinetic_terms_continuum_position}
\end{equation}
and in momentum space as
\begin{equation}
    \begin{aligned}
        \widetilde{K}(p, q) &=  \int_{x,y} e^{-ipx} \left(\partial^2_x + m_\phi^2\right) \ \delta(x,y) \ e^{iqy} = \left(\frac{p^2}{2} + \frac{m_\phi^2}{2}\right) \ \delta(p,q) \\
        \widetilde{D}(p, q) &= \int_{x,y} e^{-ipx} \left(\slashed{\partial}_x + m_q + g\phi \right) \ \delta(x,y) \ e^{iqy} = \left(\slashed{p}_x + m_q + g\phi \right) \ \delta(p,q)
    \end{aligned}
    \label{eq:definition_kinetic_terms_continuum_momentum}
\end{equation}
This allows one to rewrite the action as
\begin{equation*}
    S[\phi,\psi,\bar\psi] = \int_x \frac{1}{2} \, \phi_x \, K_{xx} \, \phi_x + \frac{\lambda}{4!} \, \phi_x^4 + \sum_{f=1}^{N_f} \bar\psi_x^{(f)} \, D_{xx} \, \psi_x^{(f)}
\end{equation*}
We introduce the left-handed and right-handed spinors
\begin{equation*}
	\psi_L = (1-\gamma_5) \, \psi \qquad \psi_R = (1+\gamma_5) \, \psi
\end{equation*}
for which
\begin{equation*}
	\psi = \frac{(1-\gamma_5)}{2} \psi + \frac{(1+\gamma_5)}{2} \psi = \psi_L + \psi_R
\end{equation*}
The action written in terms of $\psi_L, \psi_R$ reads
\begin{equation}
	S = S_\phi +  \bar\psi _L D \psi_L + \bar\psi _R D \psi_R + (m_q + g\phi) \,  \left(\bar\psi_L\psi_R + \bar\psi_R\psi_L\right)
	\label{eq:action_chirality_explicit}
\end{equation}
The last equation makes clear that that for $m=0,\left\langle\phi\right\rangle = 0$ the action is symmetric under the chiral group $SU(2) _L\times SU(2)_R$, namely
\begin{equation*}
	\begin{aligned}
		\psi_L(x) \to U_L\psi_L(x) &\qquad \bar\psi_L(x) \to \bar\psi_L(x) U_L^{\dagger} \\
		 \psi_R(x) \to U_R\psi_R(x) &\qquad \bar\psi_R(x) \to \bar\psi_R(x) U_R^{\dagger}
	\end{aligned}
\end{equation*}
for $U_L, U_R \in SU(2)$.\\
The main feature of the model is chiral symmetry breaking \cite{Nambu1961DynamicalI, Nambu1961DynamicalII}, which can happen explicitly at the level of the classical action for a non-zero quark mass, or spontaneously when the scalar field gains a non-zero expectation value. One can in fact notice already by looking at \eqref{eq:definition_kinetic_terms_continuum}, that $\left\langle\phi\right\rangle \neq 0$ has the same effect on the action as a finite bare quark mass. This observation will be made more quantitative in section (SECCCC) where it will be shown that  
\begin{equation*}
    \left\langle \phi \right\rangle \sim \left\langle \bar \psi \psi \right\rangle \sim m_q
\end{equation*}
The fermionic part of the path integral \eqref{eq:path_integral_generic} can be performed explicitly
\begin{equation*}
    \int \mathcal{D} \bar\psi \mathcal{D} \psi \ \exp\left( - \int_x \sum_{f=1}^{N_f} \bar\psi_x^{(f)} \,  D \, \psi_x^{(f)} \right) = (\text{det} \, D[\phi])^{N_f} = e^{N_f\tr \log (D[\phi])}
\end{equation*}
where the trace is performed over space-time and spinor components. \\ 
The full path integral can now be expressed in terms of the resulting effective action for the scalar fields
\begin{equation*}
    Z = \int \mathcal{D}\phi \ e^{-S_\text{eff}[phi]}
\end{equation*}
with
\begin{equation}
    S_\text{eff}[\phi] = S_{\phi}[\phi] - \underset{x,s,f}{\tr} \log D[\phi]
    \label{eq:effective_action_no_fermions}
\end{equation}
One can derive the classical equations of motion by imposing $\frac{\delta S}{\delta \phi} = 0$, here expressed in momentum space
\begin{equation*}
     (k^2 + m_\phi^2) \, \phi(x) + \frac{\lambda}{6} \, \phi^3(x) = g \ \underset{s,f}{\tr} \left[D^{-1}(\phi(x)\right] = - g \ \bar\psi(x) \psi(x)
\end{equation*}
where the trace is performed over spin and flavour components. For $\lambda = 0$, they highlight a simple proportionality relation between magnetisation and chiral condensate, which for zero momentum reads
\begin{equation}
    \phi(x) = - \frac{g}{m_\phi^2} \ \bar \psi(x) \psi(x)
    \label{eq:classical_EOM}
\end{equation}
The classical relation \eqref{eq:classical_EOM} is proven to hold also at mean field on the quantum level \cite{Buballa2005NJL-modelMatter} and will be studied in the discretised theory in section \ref{sec:classical_to_quantum}. \\~\\


\begin{figure}[h]
\centering
\begin{minipage}{0.45\textwidth}
    \begin{tikzpicture}
        \begin{axis} [axis lines=center, xtick=\empty, ytick=\empty, xlabel=$\phi$, ylabel=$V(\phi)$,
        every axis x label/.style={
            at={(ticklabel* cs:1.0)},
            anchor=west,
        },
        every axis y label/.style={
            at={(ticklabel* cs:1.0)},
            anchor=south,
        },]
            \addplot [domain=-3:3, smooth, thick] { 6*x^2 + x^4 - 6*x };
        \end{axis}
     
    \end{tikzpicture}
    
\end{minipage}
\hfill
\begin{minipage}{0.45\textwidth}
    \begin{tikzpicture}
        \begin{axis} [axis lines=center, xtick=\empty, ytick=\empty, xlabel=$\phi$, ylabel=$V(\phi)$,
        every axis x label/.style={
            at={(ticklabel* cs:1.0)},
            anchor=west,
        },
        every axis y label/.style={
            at={(ticklabel* cs:1.0)},
            anchor=south,
        },]
            \addplot [domain=-3:3, smooth, thick] { -6*x^2 + x^4 - 2*x };
        \end{axis}
     
    \end{tikzpicture}
\end{minipage}
\label{fig:breaking_O1_symmetry}
\caption{The introduction of the boson-fermion interaction, with a finite fermionic mass, causes the breaking of the O(1) symmetry. It shifts the equilibrium position in the symmetric phase (left) causing $\left\langle \phi \right\rangle \neq 0$, and tilts the potential in the broken phase (right), making the two minima not equivalent.}
\end{figure}

\newpage
